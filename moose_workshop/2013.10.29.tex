\section{Intro}
Jean Ragusa introduces Derek Gaston. Derek started the MOOSE project with Rich 
Martineau. 



\section{Framework}

MOOSE started in May 2008. Leverages petsc and libMesh ( ) among other things. 


\subsection{Coupling}
\begin{itemize}
  \item Loose Coupling / Operator Split
    \begin{itemize}
      \item Solve PDE1
      \item Pass Data
      \item Solve PDE2
      \item Move to next timestep
    \end{itemize}
  \item Tight Coupling 
    \begin{itemize}
      \item Solve PDE1
      \item Pass data
      \item Solve PDE2
      \item Pass Data
      \item Solve PDE1
      \item Pass Data
      \item Solve PDE2
      \item Converge
      \item Move to next timestep
    \end{itemize}
  \item Full Coupling / JFNK
    \begin{itemize}
      \item Solve PDE1 and PDE2 simultaneously
      \item Move to next timestep
    \end{itemize}
\end{itemize}

\subsection{Strong Form}
The normal, dimension agnostic form of any differential equation. You should 
write it out dimension-agnotstically because you want to be able to solve it for any 
number of dimenstion.

\begin{align}
  \rho C_p \frac{\partial T}{\partial T}- \grad \cdot k(T,B)\grad T = f
  \label{strongform}
\end{align}

\subsection{Weak Form}

The first and only step of getting this into MOOSE is to get the strong form 
into the weak form. This is the integral form, and operates on the test function 
$\psi_i$.

The actual code looks very simple, and translates exactly from the weak form 
into symbolic code. 


\subsection{Pluggable Interfaces}

There are many physics modules already built up. 

Large collections of common physics include elk (extended library of kernels), fox, yak (yak enables critter, pronghorn, rattlesnake)

Applications include include bighorn, condor (superconductivity), dingo, falcon (geomechanics), ferret (ferritic oxide stuff), grizzly, heron, hyraz, mallard, mamba, 
osprey, rat, sparrow.

\subsection{SQA}
Multiple releases per day with continuous integration. Each change is vetted 
against a large test suite. 
Issue management in Trac. 
Code metrcs are calcuated. 
Documentation is automated. 

This has been NQA-1 Assessed, which means it meets all requirements to build 
nuclear safety software. 


Validation of the framework is quite complete. However, validation of the 
sub-physics are sortof up to the physics developers. 







To build the pdf that makes the docs, checkout the code. 

\begin{verbatim}
  cd papers/moose_workshop
  make
  open main.pdf
\end{verbatim}

\section{Finite Elements the MOOSE Way}

For the example to show how to find coefficients to functions, he uses a 
polynomial fitting problem. He has some data and wants to fit a, b, and c. 

\begin{align}
  f(x) = a+bx+cx^2 +\cdots
  \label{polynomial}
\end{align}

More generally, you look for 
\begin{align}
  f(x)\sum{i=0}c_ix^i
  \label{lin_poly}
\end{align}

You then take the x and y $(f(x))$ points and plug them into the linear system. 
You then use some magic method to solve the linear system and get a solution 
vector. 

\subsection{Weak Form}
Using the finite element method, you need to have the weak form. This may be 
referred to as forming a weighted resuldual or a variational statement.

Any time you use MOOSE, the first thing you need to do is generate the weak 
form. 


\begin{itemize}
  \item Write down the strong form of the PDE
  \item Rearrange terms so that zero is on the right of the equals sign (residual form)
  \item Multiply the whole equation by a ``test'' function ($\psi$)
  \item Integrate the whole equation over the domain ($\Omega$) 
  \item Integrate by parts (use divergence theorem) to get the desired 
    derivative order (differentiability)
  \item code up with moose and run
\end{itemize}

Just as a refresher, recall that the divergence theorem transforms a volume 
integral into a surface integral.  

You'll have to use it a lot. 


\subsubsection{Advection (Convection) Diffusion}

The second order derivative part is the diffusion part. 
The single order derivative part is the advective part. 

The $\vec{\beta}$ vector (obv.) is the velocity. 

\subsubsection{Shape Functions}

Shape functions are the functions that get multiplied by the coefficients and 
added up to form our solution. Different shape functions and families of 
functions.

Lagrange shape function families are the most common. They are 
interpolary at the nodes (that is, the coefficient corresponds to the value of 
the function at the node). BUT, we're still solving for a function, not a single 
vector of coefficients!

\subsubsection{Numerical Integration}

The final piece is that we do a quadrature approximation of the integral. 

\subsubsection{Newton}

We traditionally use Newton to make the update and find the converged solution 
to this nonlinear system of equations. It has good convergence properties.  

You basically start off with a guess, $x_n$. you evaluate the gradient, and 
modify the guess by adding the gradient. Lather, rinse, repeat. 



\subsubsection{Jacobian}
Usually after this, you have to find the Jacobian. This object is usually 
complicated and gigantic. 

\subsubsection{Jacobian Free Newton Krylov}

We can use a Krylov subspace method such as GMRES (or similar) to avoid the 
Jacobian. In particular, we only need to understand the action of the Jacobian 
on the vector, but never have to know the Jacobian itself. To do this, we need 
to be comfortable making the approximation :

\begin{align}
  \boldsymbol{J}\vec{v}\approx \frac{\boldsymbol{R}(\vec{u}+\epsilon\vec{v}) - 
  \boldsymbol{R}(\vec{u})}{\epsilon}.
  \label{diffapprox}
\end{align}

Note, GMRES is the work of Saad and Shultz, 1986. It's probably best to actually 
cite that work. 

\section{MOOSE Framework}

\subsection{Dependencies}
\begin{itemize}
  \item MPI (they prefer MPICH2)
  \item Hypre (from LLNL. optional, if you want to use AMG)
  \item PETSc
  \item libMesh (a vetted version comes with MOOSE)
\end{itemize}


They maintain binaries of all of these dependencies for OSX or modern ubuntu. 

\subsection{Input}

There are blocks for meshes and executioners, etc. The variables, shape 
families, etc. are all defined.

\subsubsection{Active}
Active is valid in any block. It is a shortcut for turning things on and off in 
the simulation. It tells MOOSE which blocks are active. If you leave it out, all 
of the blocks are active by default. Adding an active line with a particular 
subset of paramters will turn only those parameters on within the block.

\subsubsection{Mesh block}
Their favorite mesh format is exodus2. They also read abacus-cae, gmesh, and 
vtk. The $uniform\_refine=n$ option will split each element of the mesh $n$ 
times.

``Uniform refine is one way to destroy your computer.'' 

Second order allows you to read a first order mesh and promote it to second 
order as you read it. 

You can create curved geometries with higher order curved elements with a thing 
like CUBIT. It will output a mesh representing that curved object, which is then 
read into moose.

The mesh generator will allow you to name or number the boundaries. You can use 
those names to the assign boundary conditions in the input file.

\subsection{Variables}

Each variable gets its own choice of shape function. you can mix all of these 
shape functions. 


\subsection{Kernels}

You will declare the pde operators. There are already registered types that you 
can rely on which to base this. That might be a heat conduction kernel, 
diffusion kernel, whatever.

You'll specify the type of kernel, then you'll say which (already named in the 
variable block) variable on which it will act. 



\section{Cubit}

CUBIT can be licensed for CSimSoft for a fee. It's complicated even for 
universities these days. If you're working on a government contract, CUBIT is 
still free. 

Other mesh generators can work as long as they output a file format that libMesh 
reads. There are a lot of them. 

The only parallel mesh type that they read is Nemesis, a type of exodus mesh. It 
can be generated by cubit. See nemesis\_test.i in the moose\_test directory for a 
working example.


\section{Style Standard}

\begin{verbatim}
ClassName
methodName
_member_variable
local_variable
src/ClassName.C
include/ClassName.h
two spaces for indentation
four spaes for init lists
braces occupy their own line (and not be indented in crazy libmesh emacs style}
spaces around all operators and declarations
no trailing whitespace
doxygen docs for each method
\end{verbatim}


\section{Template Specialization}

The idea is that we have some main function that calls print(a) and print(b). We 
want print to print the number if it gets an int. But, when it gets a bool, we 
want it to print the string ``true'' or ``false''.

To do this, we use \verb|template<>| to describe the one .

\begin{verbatim}
template<> 
InputParameters validParams<MyKernel>(
{
  InputParameters params = validParams<Kernel>();
  params.addRequiredParam<RealVectorValue>(``paramname'', ``param docstring'');
  return params;
}
\end{verbatim}


\section{Output}

\subsection{OverSampling}
Generally available visualization tools don't display the higher order solutions 
natively. The values that are plotted must be on the mesh nodes. To see the 
higher order stuff and get a smoother solution, you can oversample, which 
involves interpolating the solution onto a higher mesh. 

This will also work with the mesh-adaptivity of moose, such that the vizualized 
output is not mesh adapted. Rather, it is refined so that the grid that gets 
vizualized is regular. 

The oversampling is not the same as solving the solution on a finer grid. It is 
interpolating the coarse-grid solution (which has continuous shape functions) 
over a finer grid.

\section{Input Parameters and MOOSE Types}

\subsection{Valid Paramters}

This is where we use the template syntax we went over above. You can 
addRequiredParam, you can addParam (which accepts a default), you can also add 
paramters of container types.

Types that you may not recognize and are MOOSEy: 
\begin{itemize}
  \item Point
  \item RealVectorValue
  \item RealTensorValue
  \item SubdomainID
  \item BoundaryID
  \item Parser::ExtractParams(\ldots) will give you a full list of these types!
\end{itemize}

\subsection{Moose Enum}
\ldots is peculiar.



\section{Kernels}
It can represent one or more operators in PDE and overrides 
computeQpResidual() which returns a weak form of the pde operator thing. 

The Kernel can optionally override the jacobian and off-diag jacobians. It's 
going to look a lot like the jacobian, but may not be the full on jacobian. It 
will be a simple version of the jacobian and will act as a precondicitoner.


All Kernels inherit from the Kernel class. They will therefore include Kernel.h 
and will have the public inheritance declartion as a derived class of Kernel.

In general, most functions should be protected.


\subsection{Member variables}
The \_u and grad\_u are the most important, because those are what the kernel is 
operating on. The name never changes when you're building a kernel object.


The quadrature point where you are right now is the ``q-point.'' There was a 
question about how we calculate quadrature points, but, importantly, we do not 
choose them. You can override the choice, but, usually, MOOSE chooses the 
quadradture points.  

The i always goes with the test function in moose the j always goes with the 
jacobian. Thus, don't screw up \_i and \_j because they refer to the test and 
trial functions respectively. 

If you need to inspect the current element, you can get that from the object. 

\subsection{Diffusion Example}

Creating this kernel object requires the residual, but it's also easy to 
calculate the jacobian as well, so we do that (because we can). 




\section{Running MOOSE}

You should use some level of j with make. 

\begin{verbatim}
make -j4 (or whatever)
\end{verbatim}

By default, you will run in optimized (opt) mode. You may prefer to work in 
oprof in order to do more extensive debugging and profiling. 

So, you can go to the moose\_examples directory and run make.

The base folder will hold your app. The src will hold your main.C. The 
executable will be created in the root directory of the example. Run it.

When it runs, it will give you lots of useful information. While timing output 
is not listed by default, you can set it somewhere. 

The out.e file can be opened in Paraview! It's pretty dang good!


If things are going slowly, you can turn on algebraic multigrids with petsc 
options (-pc\_type hypre -pc\_hypre\_type boomerang)


You can override parameters on the commandline :

\begin{verbatim}
./example -i input.i -pc_type hypre -pc_hypre_type boomerang Mesh/uniform_refine=8
\end{verbatim}


To use more processors, just throw in an mpiexec 

\begin{verbatim}
mpiexec -n 4 ./example -i input.i -pc_type hypre -pc_hypre_type boomerang Mesh/uniform_refine=8
\end{verbatim}


To use more threads (like, on a cluster with many processors per node)

\begin{verbatim}
mpiexec -n 2 ./example -i input.i -pc_type hypre -pc_hypre_type boomerang Mesh/uniform_refine=8 --n-threads=2
\end{verbatim}



\section{Coupling}

Simultaneous solves just happen in the input file when you call both diffusion 
and advection on the same param. 

But, sometimes you need to refer to an unknown variable in order to solve for 
another. 

So, the way this goes is set out in the nonlinear transient oxygen 
non-stoichiometry example. It shows that, to form the residual for a coupled 
nonlinear residual, we take the residuals for each of the physics, and then we 
stack them on top of each other in a vector.  

That was complicated though. The simpler example is one way coupled diffusion 
advection (where the derivative of the pressure, or whatever, drives flow).

You can grab the gradient from elsewhere in the simulation and use that as the 
velocity vector instead of just specifying it in the input file. 


To couple a variable into a moose object (such as a Kernel), you do this with 
the valid params functionality. This allows you to do the coupling in the input 
file. You do this with addCoupledVar. There are also variables called auxilliary 
variables that allow you to couple and uncouple things. 

Once you join the coupling parameter with your kernel, you need to call some 
functions to gain accesses to that variable. You will do this in the 
initialization list in your constructor (these funcitons return references, so 
you have to grab it and hold onto it thusly). You can get their values, their 
value from last timestep, their gradients, old gradients, older gradients, etc. 
You can also get the time derivative (coupledDot). Once you grab those 
references, you use them in computing the residual and jacobians. 





